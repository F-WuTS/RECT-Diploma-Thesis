\chapter{Study of Literature Sztavinovszki}

\textbf{Author: Jeremy Sztavinovszki} 

\section{The Rust Book}
% summary and main takeaway


\section{Network Programming with Rust}
% summary and main takeaway



\section{Main Takeaways}

\section{Getting Started with Bluetooth Low Energy}
Getting Started with Bluetooth Low Energy\footcite{ble_book} is a book giving deep insight into
the inner workings of Bluetooth Low Energy. The first four chapters are especially useful to this
work and are summed up in the following sections.

\subsection{Chapter 1. Introduction}
The introduction explains the history of Bluetooth Low
Energy from its start as Wibree, over the adoption by the Bluetooth Special Interest Group, to
the status at the time of the writing of the book. The chapter then goes on to explain the difference
to Bluetooth Classic. Key Limitations like Data througput and Operation range as well as possible network
topologies and the difference of Protocols versus profiles are also explained.  

\subsection{Chapter 2. Protocol Basics}
The second chapter goes over the different layers and protocols of Bluetooth Low Energy explaining the importance
and use for each of the layers and also covering the Generic Attribute Profile and the Generic Access Profile. It
provides explanations for the frequency hopping and modulation done in the physical layer and many other specifics
in the other layers.

\subsection{Chapter 3. GAP(Advertising and Connections)}
The third chapter explains concepts like roles, such as Broadcaster and Observer, and explains how each of them work.
It gives explanations of the modules and procedures included in the General Access Profile. After that section the Security
Manager and General Security constructs, such as Address Types, Authentication and Security Modes are covered. Lastly the GAP
service and the format in which data is sent over advertisements is covered.

\subsection{Chapter 4. GATT(Services and Characteristics)}
This chapter goes on to explain the roles defined in the Generall Attribute Profile and covers UUIDs.
Then the most important concept, Attributes, are explained. This chapter along with the documentation for the library this work uses for BLE solidified, that L2CAP is more suitable for this work. 
% Bla bla bla. Hier weiter schreiben.

\subsection{The Remaining Chapters}
The remaining covers go over specific hardware platforms, debugging tools, application design tools and platform specific mobile
programming and were not as important as the previous four chapters. Therefore they are not covered here.

\subsection{Main Takeaways}
The book provides great explanations of many difficult topics through text as well as visual aids. The main takeaways for this work are, that L2CAP is better suited for its purposes and better general knowledge of the BLE and its stack.


\filbreak
