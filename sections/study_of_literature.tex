\tocdata{toc}{$\rightarrow$\textit{Jeremy Sztavinovszki}}
\chapter{Study of Literature}
\textbf{Author: Jeremy Sztavinovszki} 

\section{The Rust Programming Language}
% summary and main takeaway
"The Rust Programming Language"\footcite{the-rust-programming-language}, as the name would suggest is the official book for learning rust. It is a great resource for learning the Rust language
and is available online as a website, as well as a downloadable PDF, or a Paperback. It is divided into 20 chapters, starting off with the installation
of Rust and basic concepts like variables and control flow, and finishing by explaining how to program a web server.

\subsection{The Basics}
Chapters 1 to 9 cover the most basic concepts of Rust. They go over simple I/O programming and explain common concepts such as ownership, borrowing, and lifetimes.
They also go over common collections, how to define and use enumerations and structs and how to manage a project in Rust. The most important parts of these chapters
are the explanations of ownership, borrowing, lifetimes, and error handling, as these are the concepts most beginners tend to struggle with when learning Rust.

\subsection{Intermediate Concepts}
Chapters 10 to 14 cover more intermediate concepts such as generics, traits, and writing unit tests. Chapter 13 goes over the functional features of Rust, explaining
closures, iterators, and how to use them. These chapters are important for understanding the more advanced concepts of Rust and provide the basis for writing better
and more efficient code.

\subsection{Advanced Concepts}
Chapters 15 to 20 cover the more advanced conecepts of Rust. They go over how to use unsafe code, how to use concurrency, smart-pointers, and many more advanced
features of Rust. Chapter 15 covers smart-pointers and why they are useful for memory safety and concurrency. Chapter 16 goes over how to use concurrency in Rust
covering the basics of how to start a new thread and then going over how to use message-passing and shared-state concurrency. Chapter 17 covers the object oriented
programming features of Rust, explaining how to use traits and how to use them to implement object oriented programming. Chapter 18 goes over patterns and how to use
them as a way control the flow of a program and where they should be used. Chapter 19 goes over the advanced features of Rust, such as macros, unsafe Rust, and advanced traits
and types. Chapter 20 is the final chapter of the book providing a kind of final project by first building a single threaded web server, then making it multi-threaded and 
finally going over graceful shutdown and cleanup of the resources the server used.  

\subsection{Takeaways}
"The Rust Programming Language" is a great resource for learning Rust. It provides a great introduction to the language and then gradually goes over more advanced topics.
RECT uses many of the concepts explained in the Rust book such as concurrency and error handling, which makes the knowledge gained from the book very useful.

\section{Getting Started with Bluetooth Low Energy}
"Getting Started with Bluetooth Low Energy"\footcite{ble-getting-started} is a book giving deep insight into
the inner workings of Bluetooth Low Energy. The first four chapters are especially useful to this
work and are summed up in the following sections.

\subsection{Chapter 1. Introduction}
The introduction explains the history of Bluetooth Low
Energy from its start as Wibree, over the adoption by the Bluetooth Special Interest Group, to
the status at the time of the writing of the book. The chapter then goes on to explain the difference
to Bluetooth Classic. Key Limitations like Data througput and Operation range as well as possible network
topologies and the difference of Protocols versus profiles are also explained.  

\subsection{Chapter 2. Protocol Basics}
The second chapter goes over the different layers and protocols of Bluetooth Low Energy explaining the importance
and use for each of the layers and also covering the Generic Attribute Profile and the Generic Access Profile. It
provides explanations for the frequency hopping and modulation done in the physical layer and many other specifics
in the other layers.

\subsection{Chapter 3. GAP(Advertising and Connections)}
The third chapter explains concepts like roles, such as Broadcaster and Observer, and explains how each of them work.
It gives explanations of the modules and procedures included in the General Access Profile. After that section the Security
Manager and General Security constructs, such as Address Types, Authentication and Security Modes are covered. Lastly the GAP
service and the format in which data is sent over advertisements is covered.

\subsection{Chapter 4. GATT(Services and Characteristics)}
This chapter goes on to explain the roles defined in the Generall Attribute Profile and covers UUIDs.
Then the most important concept, Attributes, are explained. This chapter along with the documentation for the library this work uses for BLE solidified, that L2CAP is more suitable for this work. 
% Bla bla bla. Hier weiter schreiben.

\subsection{The Remaining Chapters}
The remaining covers go over specific hardware platforms, debugging tools, application design tools and platform specific mobile
programming and were not as important as the previous four chapters. Therefore they are not covered here.

\subsection{Takeaways}
The book provides great explanations of many difficult topics through text as well as visual aids. The main takeaways for this work are, that L2CAP is better suited for its purposes and better general knowledge of the BLE and its stack.


\filbreak
