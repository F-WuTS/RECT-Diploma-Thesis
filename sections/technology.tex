\chapter{Technology}


\section{Wombat}

\section{Python}

\section{C++}

\section{Rust}
\textbf{Author: Jeremy Sztavinovszki}

Rust is a general purpose multi paradigm programming language used in many fields ranging from embedded programming to web development. Although it is a relatively young language, having released its version 1.0 on May 15th 2015, it has seen great adoption from developers and has a big community. The language tries to be as fast as possibly, while still remaining memory-safe, which it achieves using its borrow checker. Even though it is possible to write unsafe code in Rust, that is not checked by the borrow checker, it is custom to keep the unsafe parts as small as possible.
Rust has a great ecosystem driven by the Rust Foundation and the Rust community. There are many tools, such as cargo, or rust-gdb, that provide great developer experience.
Right now there is now standardized async-runtime, so you normally use runtimes like tokio, async-std, or smol for programming asynchronously.

\section{gRPC}

\section{Nix}
\textbf{Author: Jeremy Sztavinovszki}

Nix is one of a couple of things depending on the context. It is either a configuration language, a package manager, an operating system, or a build system. 
That may seem a bit confusing, but the next section will cover each of these contexts for the sake of clearing up some of the confusion, that may arise from this statement.

\subsection{History}
Nix came into being in 2003 as a research project by Eelco Dolstra. At first it was just a package manager, that could be run on any distro, but in 2007 it became its 
own full blown Linux distribution with many other additions to the Nix eco system, such as Hydra \footcite{hydra}, a continuous intergration tool, and nixops, 
a deployment tool for nixos deploying NixOS in a network/cloud \footcite{NixOps}

\subsection{The Language}
The language was the first part of Nix, that was implemented and it is arguably the most important part of Nix. Nix is a declarative functional programming language,
that is used for defining packages, build processes and configurations for a host of things ranging from reproducible development environments to IT-Infrastructure.
Taking an example for the syntax from the official nixos wiki site the language looks something like this: 

% example here

\begin{lstlisting}[language=Nix]
#1
let
    a = 1;
    b = 2;
in a+b

#2
let
    square = x: x*x;
    a = 12;
in square a

#3
let
    square = {x ? 2}: x*x;
    a = 12;
in square square

\end{lstlisting}


%description of the example here

\subsection{The Package Manager}
Nix, now a package manager, is a cross-platform package manager, that claims to have solved a problem called dependency hell \footcite{Dependency Hell}, by keeping track of which package needs which dependencies. If a package is no longer needed it can automatically be garbage collected. Packages are installed to a directory called the nix store and have a unique hash, which is generated by combining some factors, like dependencies, versions and so on. 
When defining a package you use the nix programming language and lazy functional programming to declare how to build it, what you need to build it and what files to install through a format called derivations.
 
\subsection{The Build System}

%build system example package here.

\subsubsection{Nixpkgs}
Of course the defined packages need to be stored somewhere. This is where the nixpkgs repository \footcite{nixpkgs_repo} on github comes in handy. It is a collection of over 80000 packages according to repology \footcite{repology_nixpkgs}. 
The community can contribute their own definitions, or updates to the repository if they found something to be out of date, or missing. Of course when installing a package it is not built from scratch every time, like on source based distros.
Instead nixpkgs caches builds of the most popular packages, which then just have to be downloaded onto the users machines. 
  

\subsection{The Operating System}

\section{Bluetooth Low Energy}
\textbf{Author: Jeremy Sztavinovszki}

Bluetooth Low Energy (BLE) is a low powered, low cost, low bandwidth radio communication technology, that was originally developed at Nokia in a project named Wibree. It was later noticed by the Bluetooth Special interest group and became part of the Bluetooth 4.0 Core Specification. Nowadays it is often used in all things ranging from wireless headphones to IOT devices and has seen great adoption in many different areas. The BLE protocol Stack, like others (e.g. TCP/IP) is separated into different layers. The layers are split into 3 overarching layers. which are the Application, Host and Controller layers.

\subsection{Application-Layer}
Much like the TCP/IP-Stack the BLE-Stack also comes with an Application-Layer. The Application-Layer is the highest layer in the stack and is responsible for containing logic, user interface and handling the data of the application using BLE. It often determines which usage model is used in the Host-Layers.
It is the layer that bundles all of the functionality of BLE together and abstracts it in such a way, that it is useable for ordinary users.

\subsection{Host-Layer}
The Host-Layer itself splits off into several layers. It is made up of all layers above the Host Side HCI, except the Application Layer, but not only that. It also contains profiles, which
determine how the protocols in the host layer should work with oneanother depending on the usage model, that has been chosen by the Application.

\subsubsection{Generic Access Profile GAP}
The Generic Access Profile, or GAP defines which role a BLE-Device has in communication. These roles determine how the device, as well as other devices act, when sending or receiving data.
A BLE device can take on 4 distinct roles, which are as follows.

\begin{itemize}
\item{Broadcaster}
\item{Observer}
\item{Peripheral}
\item{Central}
\end{itemize}


\subsubsection{Generic Attribute Profile GATT}
GATT

\subsubsection{Logical Link Protocol and Adaptation Protocol L2CAP}
L2CAP

\subsubsection{Attribute Protocol ATT}
ATT

\subsubsection{Security Manager Protocol SMP}
test

\subsubsection{Host Controller Interface HCI (Host side)}
test

\subsection{Controller}
The controller is the layer works closely with the hardware. It contains the following layers

\subsubsection{Host Controller Interface (Controller side)}
test

\subsubsection{Link Layer LL}
Hidden behind the HCI is the Link Layer. It is usually implemented as a conglomeration of custom hardware, as well as software it is the only part
of the protocol stack, that needs to have real-time capabilities, because it needs to work with the timing requirements defined by the specification.
In order to avoid overloading the CPU running the software layers of the stack, the easily automated, or computationally expensive parts are implemented in circuitry.


\subsubsection{Physical Layer PHY}
The physical layer is made up of the actual hardware, that is capable of modulating and demodulating the analog signals sent by radio and turning them into digital information. It uses the 2.4GHz ISM \footcite{ism} radio band, which it splits into 40 channels (37 for transmitting data and 3 for advertising connections and broadcasts) from 2.4GHz to 2.4835GHz. It uses FHSS to avoid radio interference, which is important, because classic bluetooth, as well as 2.4GHz use the same frequency band. The modulation rate chosen for BLE is 1.0Mbit/s, which means that is the physical throughput limit. However because of protocol overhead this physical throughput level is never reached.

\section{Libraries}

\subsection{Rusqlite}
\textbf{Author: Christoph Fellner}

Rusqlite is an ergonomic wrapper for using SQLite from Rust similar to rust-postgres. It provides an easy-to-use interface to work with SQLite databases. Using the functions provided by rusqlite you can perform all the common database operations, such as creating tables, inserting Data, querying data, and more simply within your code. 
We choose this library for SQLite mainly because of three reasons:
\begin{enumerate}
    \item \textbf{Portability:} SQLite databases can be used across different platforms and operating systems. Given the fact that RECT operates on different small Controllers, portability is very important.
    \item \textbf{Configuration:} There is no need for any complex setup or configuration when working with SQLite. We only need a few tables to work with, so having to configure a database on each controller wouldn't be worth the effort.
    \item \textbf{Local:} SQLite doesn't need any separate server or installation. It contains all the features we need in a small an independent package.
\end{enumerate}

In our case we use rusqlite in order to save the config.json file, containing date for the available connections. Accessing the Data from the database is simply much quicker and saver than reading from the file when we need it. 

\subsection{Serde}
\textbf{Author: Christoph Fellner}

Serde is a framework for rust, used for \textbf{ser}ializing and \textbf{de}serializing data structures efficiently and generically. You can find a detailed serde overview \href{https://serde.rs/}{here}.

Serde provides functions to deserialize JSON-files in a simple and quick way, this allows us to use the data from the config.json file in our program with just a few lines of code.

With serde we deserialize the data from the JSON-file into a self-made rust structure, which allows us to use the data properly.  

\subsection{Tokio}
\textbf{Author: Christoph Fellner}

Tokio is asynchronous runtime for rust. In rust asynchronous code doesn't run on it's own in order to make it work the programmer has to use a runtime like Tokio. You can find more in depth descripton of Tokio \href{https://tokio.rs/tokio/tutorial}{here}. 

We picked Tokio because it is the most widely used runtime for async rust code. There are also a lot of Tutorials for Tokio and it is fairly simple to use.

Tokio as our asynchronous runtime allows us to execute multi-threaded async code safely.  

\subsection{Tonic}
\textbf{Author: Christoph Fellner}

Tonic is a rust framework that implements gRPC.

\filbreak
