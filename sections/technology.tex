\chapter{Technology}


\section{Wombat}

\section{Python}

\section{C++}

\section{Rust}
\textbf{Author: Jeremy Sztavinovszki}
Rust is a general purpose multi paradigm programming language used in many fields ranging from embedded programming to web development. Although it is a relatively young language, having released its version 1.0 on May 15th 2015, it has seen great adoption from developers and has a big community. The language tries to be as fast as possibly, while still remaining memory-safe, which it achieves using its borrow checker. Even though it is possible to write unsafe code in Rust, that is not checked by the borrow checker, it is custom to keep the unsafe parts as small as possible.
Rust has a great ecosystem driven by the Rust Foundation and the Rust community. There are many tools, such as cargo, or rust-gdb, that provide great developer experience.
Right now there is now standardized async-runtime, so you normally use runtimes like tokio, async-std, or smol for programming asynchronously.

\section{gRPC}

\section{Bluetooth Low Energy}
\textbf{Author: Jeremy Sztavinovszki}
Bluetooth Low Energy (BLE) is a low powered, low cost, low bandwidth radio communication technology, that was originally developed at Nokia in a project named Wibree. It was later noticed by the Bluetooth Special interest group and became part of the Bluetooth 4.0 Core Specification. Nowadays it is often used in all things ranging from wireless headphones to IOT devices and has seen great adoption in many different areas. The BLE protocol Stack, like others (e.g. TCP/IP) is separated into different layers. The layers are split into 3 overarching layers. There is the Application, Host and Controller layers.

\subsection{Application-Layer}
The Application-Layer is the highest layer in the stack and is responsible for containing logic, user interface and handling the data of the actual application using BLE.

\subsection{Host-Layer}
The Host-Layer itself splits off into several layers.

\subsubsection{Generic Access Profile GAP}
GAP

\subsubsection{Generic Attribute Profile GATT}
test

\subsubsection{Logical Link Protocol and Adaptation Protocol L2CAP}
test

\subsubsection{Attribute Protocol ATT}
test

\subsubsection{Security Manager Protocol SMP}
test

\subsubsection{Host Controller Interface HCI (Host side)}
test

\subsection{Controller}
The controller is the layer works closely with the hardware. It contains the following layers

\subsubsection{Host Controller Interface (Controller side)}
\subsubsection{Link Layer LL}
\subsubsection{Physical Layer PHY}

\section{Libraries}

\filbreak
