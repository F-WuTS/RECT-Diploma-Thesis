\chapter{Experiment 1}
\textbf{Author: Timon Koch} 

In an ideal world, a communication method should provide a reliable and efficient means of transmitting data.
In reality, either reliability or throughput has to be prioritized.
To determine the best method of communication for a specific need and identify an all-rounder, two tests will be conducted.
Performance and reliability are determined by testing under optimal and more realistic conditions.
Both tests use an echo server that sends back any data sent to it.
As many packets as possible are exchanged for 60 seconds.
The packets contain a 64-bit sequence number in the first 8 bytes and are then filled with random data until the package reaches the size of the maximum transmission unit.
During this process, the time at which the packet is sent and received is recorded.
If the packet fails to return, it is considered lost.
Additionally, packets that become corrupted during transit and do not contain a valid sequence number also contribute to packet loss.

\subsection{Benchmarking Throughput and Reliability in an Optimal Environment}
To ensure optimal test conditions, two Wombats are placed about 240~centimeters apart from each other.
The experiment is conducted in a room with thick concrete walls and no other devices to prevent interference and ensure accurate results.
The Wombat's Wi-Fi module operates on the 2.4~GHz frequency band and uses a channel width of 20~MHz. One Wombat acts as a Wi-Fi access point, and after the other Wombat connects to it, both the TCP and UDP benchmarks are run.
Then, the Wi-Fi interface on both Wombats is turned off and the BLE tests are run using L2CAP.

\subsection{Testing the Effects of Interference on Throughput and Reliability}
Botball events often have many participants, which lead to interference when multiple devices broadcast using a shared medium.
Interference occurs when signals collide, corrupting the data sent.
While Wi-Fi and BLE have methods to counteract these errors, they come at a cost.
To measure the effects of interference on Wi-Fi and BLE, we simulate it.
This requires either many clients or a large antenna to saturate the entire signal spectrum.
Due to the availability of Wi-Fi and BLE-enabled devices, such as the Wombat, the former method is more practical and cost-effective.
Each device records sent and received data over time, providing insight into transfer rates and packet loss experienced by the devices.

\subsection{Results}
In ideal circumstances, Wi-Fi shows to have superior throughput and lower latency compared to BLE, making it the preferred choice for tasks that require high data rates or time-sensitive operations.
Both the TCP protocol and the BLE protocol exhibits negligible packet loss, highlighting their reliability for transmitting critical information despite BLE's lower data transmission capabilities.

Under conditions with realistic interference levels, TCP demonstrates resilience by maintaining packet integrity.
In environments where interference is prevalent, TCP over Wi-Fi is a reliable method of communication.
While BLE has limited throughput, it maintains reliability, making it useful in scenarios where energy efficiency and minimal data transmission are important.
In Botball, where the amount of energy used for Wi-Fi can be neglected, it is recommended to not use BLE to transmit data.
Since BLE uses the same 2.4~GHz bands as Wi-Fi it is also affected by interference.
Because of the high amount of interference at tournaments, it is critical for teams to create backup plans for when data cannot be transmitted successfully.

For participants in the ECER, the findings encourage teams to carefully weigh the benefits and drawbacks of implementing bot-to-bot communication in their competitive designs.
Specifically, teams can discern the optimal protocols to employ based on the specific requirements of their Botball strategies and the likely environmental conditions during the conference.
This anticipation for interference is vital for enhancing the reliability and effectiveness of their robots in the dynamic environment of robotics competitions.

This study underlines the significance of considering environmental factors such as interference when designing and testing robot communication systems.
As Botball teams strive for precision and efficiency in their robotic endeavors, the insights derived from this research offer a pathway to more effective and reliable robot-to-robot communication, ultimately enhancing the overall performance in competitions.
While this study delineates the comparative efficacy of Wi-Fi and BLE in Botball, future research should explore the integration of adaptive protocols that could dynamically switch between these methods based on environmental conditions.
\filbreak