\chapter{Conclusion}

\textbf{Author: Jeremy Sztavinovszki} 
The previous sections of the thesis went over the goals, plans for using technologies and plans for combining the parts implemented in the thesis.
This section will go over the results of the thesis, revisit the goals and plans and give an outlook on what could be done in the future.

\section{Recap}
When starting the thesis, the goal was to create a system for robot communication that is extensible to any language and resource efficient. The system was based on gRPC for the communication between programs
and used Rust to communicate with peers over Bluetooth and WiFi. The plan was to implement two libraries in popular programming languages, one in Python and one in C++, that would implement a simple abstraction,
which allows for easy communication with the Rust service running in the background. Through these libraries the user would be able to send messages to other peers and receive responses, as well as sending streams to multiple
other peers. The Rust service would handle the communication with the peers and the user would not have to worry about the underlying communication protocol. The gRPC interface for the Rust service would keep a record of
the named connections, which users defined, in an in-memory database, which the underlying protocol would use to find the correct connection to send messages to.

\section{Results}
\subsection{The Python Library}
The Python library was implemented as planned and tested with a mock implementation of the Rust service. The library provides an easy to use interface for the user to send messages and streams to other peers and receive messages and streams back.
All functionality required for the library to work as discussed in the previous sections was realized.

\subsection{The C++ Library}
Sadly due to hardship encountered with the gRPC++ library, which is described in the previous sections, the C++ library did not get implemented fully. Everything except communicating over gRPC was designed and implemented.
This also means that the Rust service was not tested with the C++ library and no results regarding memory and CPU usage can be given at this point.

\subsection{The Rust Interface}
The Rust Interface, which handles the gRPC communication with other processes and manages the connections in the in-memory database, was implemented as planned. The interface was tested through tokio tests and mock clients and worked as expected.

\subsection{The Rust Communication Library}
Although the basic functionality of the Rust Communication Library(comm\_lib) was implemented, the library is unfinished. This is due to the complexity of the concurrency model. While the library can send and receive data over WiFi and
BLE the libraries higher level abstractions like sending messages and receiving acknowledgements, sending and receiving streams and sending and receiving requests and responses have not been implemented. This means, that neither the
Python Library, nor the Rust Interface could be tested for robot to robot communication.

\section{Outlook}
While some of the parts of the RECT stack were able to be finished others encountered problems, that lead to the team unable to fully finish the implementation during the time of the thesis.
Of course after such setbacks it is important to look at what could be done better in the future. Starting with the conception of the system, the team should have looked more into existing
systems and libraries and how they solve the problems encountered during the thesis. Furthermore a more detailed plan of how to do the communication between the parts of the system should have been made
before starting with the implementation and specification of the systems interface. This would have made it easier to see the problems with the gRPC++ library and the concurrency model of the comm\_lib.
Future work on the RECT stack should mainly focus on finishing the comm\_lib and the C++ library, as well as implementing more optimizations for the communication like doing compression of the data, which is sent over the network.
Furthermore implementing more programming language interfaces would be beneficial, as it would make the system more accessible to more users. Implementing interfaces for serial communication and other communication protocols would
also be beneficial, as it would make the system more versatile. Also while a system with fixed configurations like the one implemented in the thesis may be more efficient, a system for automatically discovering peers would be easier
to use, especially in a setting, where IP-Addresses and Ports are not known beforehand. For this reason implementing a DDS system similar to the one used in ROS would be a good idea. Finally after the system is finished it should
of course be tested under real conditions, like a competition.
% Get C++ library running
% Implement more optimizations
% Implement more programming language interfaces
% Implement abstractions for more communication protocols
% Implement a dds similar to ROS DDS'
% Try the system in real competitions

\filbreak