\chapter{Tests}

\section{Benchmarking the Components of RECT}
\subsection{Setup}
\subsection{The RECT gRPC Server}
\subsection{The RECT Python Client Library}
\subsection{The RECT CommLib Library}
\subsection{Communication through RECT}

\section{Specialized Benchmarks and Tests}
\subsection{Serialization and Compression}

\subsection{Throughput and Latency for the Interfaces}

\tocdata{toc}{$\rightarrow$\textit{Christoph Fellner}}
\subsection{Database Benchmarks}
\textbf{Author: Christoph Fellner}

RECT uses a register to store data about available connections and useable ports of other controllers. This register is stored in a in-memory database. In order to find the 
best database solution for our use case, we compare SQLite, PostgreSQL and MySQL. Using docker as testing environment we can easily compare the different databases.

In order to test the databases without any influences from outside, we used docker container for each database and their corresponding client program. The client program
is written in rust and uses the corresponding rust librarys for the individual databases. However since SQLite is a file based database, we just used a volume to store the
database file, together with the rust program in one container. The rust program connects to the database and executes the queries. The program measures the time it 
takes to execute the query and prints it to the console. Because we are using docker, we can easily track the resource usage of the database container during the query. 

\filbreak