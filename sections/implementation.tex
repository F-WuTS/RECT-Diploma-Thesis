\chapter{Implementation}

\section{CommLib}
\textbf{Author: Jeremy Sztavinovszki} 
The Communication Library, or CommLib for short is the part of the RECT stack, that handles all of the communication between the hosts over traditional protocols, like TCP, UDP and BLE.
This requires it to be especially performant. In order to avoid premature optimization however, the first part of this section on the implementation of the CommLib will only cover the first versions
of the code written to get the Library to work. After the first implementation there will be benchmarks and some profiling, in order to get a grasp on which aspects of the library need to be
optimized. The second section will then cover how these results were incorperated into designing a more polished version of the CommLib.

\subsection{Setting up the Library} 
The first steps of setting up the library are more or less the same as in any other rust project. First the project is initialized with \verb+cargo new --lib <rust-name>+. This creates a
new folder with the name specified in \verb+<library-name>+ and generates some files like Cargo.toml and src/main.rs. After this step is done the needed libraries for RECT are added to the
project through \verb+cargo add <dependency-name> -F <dependency-name>/<feature-name>+ these dependencies are the pulled and built by cargo (Rust's build tool) upon the initial build of the
project. The first iteration of the project then had the following dependencies:

\begin{itemize}
	\item{tokio}
	\item{bluer}
	\item{anyhow}
\end{itemize}

All in all the commands used to generate the CommLib project and install all dependencies looked like this:
\newline
\begin{minipage}{\textwidth}
	\begin{lstlisting}[language=bash, caption=Setup Commands for CommLib]
		cargo new --lib comm_lib && cd comm_lib
		cargo add tokio bluer anyhow -F tokio/full,bluer/full
		cargo build
	\end{lstlisting}
\end{minipage}

\subsection{First Implementation}
The first part of the implementation that was tackled was to create an abstraction layer over the existing protocols that RECT uses.
in order to have a nice and clean interface to work with and to avoid having to implement each feature separately for the
protocols. Of course, there was a bit of a problem with UDP because it is not meant to send structured data, so there is no feature parity between TCP and BLE.
between TCP and BLE and UDP in this respect, and UDP is only used to send unstructured data streams. To encapsulate the structured and unstructured data sent over the
and unstructured data sent over the common interface, there needs to be a way to convert the data, whether structured or not, into and from bytes in a way that is performant and has minimal overhead.
and has minimal overhead. In order to meet the above requirements, the following structure has been implemented.
 
\subsubsection{Messages and Packets}
% TODO Write about the packets and messages and maybe do a nice graphic.
% TODO find cite for TCP and UDP MTU 
The first thing taken into consideration for designing a data and class structure that is able to be sent over all of the protocols is the MTU's of the different protocols. 
For TCP and UDP the MTU, or maximum transmission unit, is defined by the Maximum Segment Size Option (MSS), which is technically limited to 65535 bytes (64KB), but as defined
in RFC 2675 \footcite{rfc2675} an MSS value of 65535 is defined to be interpreted as infinity and to be determined by Path MTU Discovery \footcite{rfc9293}. For BLE the MTU is 
defined by the L2CAP and can be anywhere from 23 and infinity, but the packet is fragmented and recombined by the L2CAP for transmission. % TODO find out why ble book cant be cited 


% TODO Write about the connection manager and also program that stuff
\subsubsection{The ConnectionManager}

% TODO Pretend you didn't already make everything railway programming and it still crashes because of stupid things.

\subsection{Profiling and Benchmarking}

\subsection{Polishing}

\subsection{Documentation}

\section{Rust Service}
\subsection{Documentation}

\section{C++ Implementation}
\textbf{Author: Maximilian Dragosits}
The C++ Implementation is one of the two outward facing components of the RECT stack. Alongside the Python Implementation 
it serves as a library in order for developers to be able to create robots, that are able to communicate with each other, much
easier then before. This is accomplished by abstracting most of the complexities of gRPC behind the \textit{Rectcpp} class. 

The class only needs to be initialized with IP-Addresses for the different services that it offers and be given the IP of 
another of its kind and then it should be a simple act of using the predefined methods within the class in order to 
effortlessly communicate with other robots or devices running this or the Python frontend implementation.

\subsection{Rectcpp class}
\subsection{Documentation}

\section{Python Implementation}

\subsection{Documentation}

\section{Implementation Comparison}

\filbreak
