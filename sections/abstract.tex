

\tocdata{toc}{$\rightarrow$\textit{Jeremy Sztavinovszki}}
\chapter{Kurzfassung}
\textbf{Author: Jeremy Sztavinovszki}

\vspace{10mm}
Diese Arbeit beschreibt, wie ein Kommunikationsprotokoll und Bibliotheken für Robotik 
implementiert werden können. Sie untersucht verschiedene Implementierungsmöglichkeiten 
und versucht, mögliche Schwierigkeiten und anwendbare Optimierungen herauszufinden, um
eine Methode für die Kommunikation in der Robotik bereitzustellen. Es gibt bereits mehrere 
Kommunikationsprotokolle und Bibliotheken für die Robotik, jedoch sind sie entweder auf eine
bestimmte Sprache spezialisiert oder rechenintensiv auf, der ihre Verwendung 
auf Plattformen mit geringer Rechenleistung erschwert. Diese Arbeit beschreibt die Entwicklung
eines erweiterbaren Kommunikationsprotokolls sowie die Implementierung von Bibliotheken dafür
in verschiedenen gängigen Sprachen. Sie beschreibt zunächst den Planungsprozess und 
präsentiert dann die allgemeine und Netzwerkarchitektur sowie Tests und Benchmarking für
die Leistung und Ressourcennutzung der verschiedenen Teile der entwickelten Kommunikationsmethode.
Die Arbeit stellt ein vielseitiges und effizientes Kommunikationsprotokoll für die Robotik vor,
das bestehende Einschränkungen adressiert und zu Fortschritten in diesem Bereich beiträgt.

\tocdata{toc}{$\rightarrow$\textit{Jeremy Sztavinovszki}}
\chapter{Abstract}
%\addtocontents{toc}{\textit{Jeremy Sztavinovszki}\par}
\textbf{Author: Jeremy Sztavinovszki}

\vspace{10mm}

This thesis describes how to implement a communication protocol and libraries for robotics. 
It explores the different possibilities for implementation and seeks to find out possible 
difficulties and applicable  optimizations in order to provide a method for communication 
for robotics. There are already  quite a few communication protocols and libraries for 
robotics available, however they are either specialized to a certain language, or have an 
overhead, that makes them difficult to use on low power platforms. This thesis describes
the development of an extensible communication protocol as well as implementing libraries 
for it in different popular languages. It first describes the planning process and then
goes on to showcase the general and network architecture, as well as testing and benchmarking 
the performance and resource-use for the different parts of the developed communication
method. This thesis presents a versatile and efficient communication protocol for robotics,
addressing existing limitations and contributing to field advancements.


