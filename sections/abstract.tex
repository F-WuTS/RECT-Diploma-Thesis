\chapter{Kurzfassung}

\textbf{Author: Sztavinovszki}

\vspace{10mm}

Eine der am meisten in der Robotik verwendeten Software Suiten ist das sogenannte Robot Operating System \footcite{ros-site} (kurz ROS).
Es ist eine Sammlung von Werkzeugen und Packeten, die verwendet werden, um hoch performante Robotik Systeme zu entwickeln und es abstrahiert
die Kommunikation zu Topics und Messages, die man in ROS-Bag Files aufzeichnen und abspielen kann. Das erlaubt es entwicklern sich auf die tatsächlichen Probleme der Entwicklung zu konzentrieren und sich nicht darum kümmern zu müssen Kommunikation neu zu implementieren.
ROS erlaubt außerdem die simulation Zahlreicher Komponenten von Robotern, welche man meistens als Digitale Zwillinge bezeichnet.
Einer der größten Nachteile, die ROS mit sich bringt ist die Komplexität und der Eigensinn, bei der Entwicklung. Diese Komplexität führt außerdem zu Performance einbußen, welche vor allem die Verwendbarkeit auf weniger Leistungsstarken Geräten, wie zum Beispiel dem KIPR Wombat Controller \footcite{wombat-controller}, welcher bei Schülern wegen seiner Kostengünstigkeit und Verwendbarkeit zum Einsatz kommt.

\medskip

Diese Arbeit wird erkunden, wie man eine Software ähnlich zu ROS implementiert. Es werden Probleme, wie zum Beispiel Inter Prozess Kommunikation \footcite{ipc-begriff}, Kommunikation durch Bluetooth Low Energy (BLE), TCP und UDP, und die Implementation von Kommunikations Libraries für verschiedene Programmiersprachen


\chapter{Abstract}

\textbf{Author: Sztavinovszki}

\vspace{10mm}


% Context or background information; general topic; specific topic.
In robotics one of the most important topics over the last couple of years has been communication. Communication doesn't only concern a robot being remote controllable. Mimimi keine Ahnung. 

% central questions or problem statement

% what's already known, what previous research has shown

% main reasons, rationale, goal for research

% research and or analytical methods

% main findings, results or arguments

% significance or implications

