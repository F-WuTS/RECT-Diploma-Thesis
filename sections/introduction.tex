\chapter{Introduction}

\makeatletter
\renewcommand\paragraph{\@startsection{paragraph}{4}{\z@}%
% display heading, like subsubsection
                                     {-3.25ex\@plus -1ex \@minus -.2ex}%
                                     {1.5ex \@plus .2ex}%
                                     {\normalfont\normalsize\bfseries}}
 \setcounter{secnumdepth}{4}
\makeatother

\vspace{2mm}

\tocdata{toc}{$\rightarrow$\textit{Jeremy Sztavinovszki}}
\section{Motivation}
\textbf{Author: Jeremy Sztavinovszki}

In almost every robotics application nowadays you need some kind of communication. Wether it is a robot communicating
its data to a home-base, or two robots sharing data with one another. Over the past years communication has drastically
improved with new protocols and technology, such as Bluetooth Low Energy(BLE) \footcite{bluetooth-low-energy}. On the other hand there are still people, that
don't use any of the communication technologies described before, because most of the established frameworks seem too
complex or have too many requirements regarding performance. 

\tocdata{toc}{$\rightarrow$\textit{Jeremy Sztavinovszki}}
\section{Goal}
\textbf{Author: Jeremy Sztavinovszki}

This diploma thesis will take explore how to write a communication framework using new technologies, such as BLE with the Transmission Control Protocol(TCP) \footcite{rfc9293}
and the User Datagram Protocol (UDP)\footcite{rfc768}.

At the end of the project the framework should be useable for sending data between two KIPR Wombat-Controllers \footcite{wombat-controller}. It should be capable
to send well structured data, e.g. a JSON\footcite{json}-file, as well as streaming data, e.g. a series of pictures, between the robots. When using protocols, that don't make guarantees about
the completeness of the received data RECT will not provide extra safeguards to make sure all sent data is also received.
RECT should be able to provide a stable and fast connection at reasonable distances for the selected protocol (e.g. circa 30 meters of line-of-sight, for BLE).
RECT should provide libraries for the languages Python and C++ that hide the complexity of communication and provide nice abstractions simplifying its use.
For this to be made possible the following requirements have to be met:
\begin{itemize}
\item A Rust library for communicating via TCP, BLE and UDP has to be written.
\item A Rust gRPC service using the communication library has to be implemented.
\item A Python and a C++ gRPC client and wrapper library have to be implemented.
\end{itemize}

\tocdata{toc}{$\rightarrow$\textit{Maximilian Dragosits}}
\section{History}
\textbf{Author: Maximilian Dragosits}
\subsection{Networks}
The history of communication and coordination between seperate systems is long and extensive. Nowadays it is an essential 
function of almost every technological device. In the field of computing it started in the late 1950s with SAGE (Semi-Automatic Ground Enviornment),
which was a network of computers and networking technology created by the United States of America military, 
and it allowed the transfer of radar data nation-wide.\footcite[][89]{A_New_History_of_Modern_Computing}\\
The next major step forward was the beginning of ARPANET in 1969, which served as a connection between multiple north american 
universities, and laid the groundwork for the modern internet.\footcite[][25]{How_the_web_was_born}
In more recent times the Internet of Things is commonplace, and is used for the interchange of terabytes of data each day. 
But there are also many smaller private networks used either for simple processes or sensitive data, that shouldn't be 
accesible by a theoretical viewer outside the trusted circle. 

\subsection{Protocols}
In the current network communication landscape, various protocols enable the smooth transfer of data. One of these protocols is the 
Transmission Control Protocol (TCP), which has become the dominant choice among Transport layer protocols due to its widespread use on the internet. 
The TCP/IP protocol, which was first described in RFC 675\footcite{rfc_675}, is the foundation of modern network communication. It establishes a robust framework that 
underpins the vast expanse of the internet. Alongside TCP, the User Datagram Protocol (UDP), conceived by David Patrick Reed in 1980, is a formidable companion. 
These protocols, TCP and UDP, have solidified their positions as stalwarts in data transfer methodologies across the internet and analogous networks.\\

The introduction of TCP/IP was a significant moment in the development of network communication. RFC 675 established the basic principles that continue 
to influence modern internet protocols. UDP, although different in design and use, complements TCP by offering a lightweight option for situations where 
real-time data transmission is more important than guaranteed delivery. TCP, UDP, and IP collectively form the Internet Protocol Suite, which governs the 
functioning of the internet and facilitates global connectivity. Their enduring significance in shaping network protocols underscores their indispensable role 
in our interconnected digital world.

\subsection{Bluetooth}
Establishing seamless communication between multiple mobile and fixed devices in close proximity, along with the formation of Personal Area Networks (PANs), 
has traditionally presented considerable challenges. To address this issue, Bluetooth technology emerged as a groundbreaking solution. The genesis of this 
innovative concept can be traced back to 1989 when Ericsson Mobile envisioned a wireless solution for headsets. Following the conceptualization phase, active 
development began in 1994. In 1997, IBM collaborated with Ericsson to integrate Bluetooth technology into the IBM ThinkPad, recognizing its potential. 
This collaboration resulted in the incorporation of Bluetooth functionality into both the ThinkPad notebook and an Ericsson mobile phone, marking a pivotal moment 
in the evolution of wireless connectivity.\\

In 1999, the first device equipped with Bluetooth functionality, a wireless headset, marked a significant leap forward for the technology. This milestone 
revolutionised the way devices communicate over short distances and laid the foundation for the widespread adoption of Bluetooth across a myriad of electronic 
devices, ranging from smartphones to smart home gadgets. The evolution of Bluetooth from its origins as a wireless headset solution to its current status as a 
crucial element of various technological ecosystems highlights its lasting influence in enabling effortless connectivity in both personal and professional contexts.

\tocdata{toc}{$\rightarrow$\textit{Jeremy Sztavinovszki}}
\section{Project Management}
\textbf{Author: Jeremy Sztavinovszki}


The obvious choice for working on a group project is using SCRUM \footcite{what-is-scrum}, because it allows complicated projects to be completed in a fashion, 
that is able to adapt to new requirements by for example customers. One of the biggest downsides of SCRUM the members of the project saw was the high overhead of planning sprints. 
For that reason Kanban was chosen for managing the tasks, that need to be done in order to finish the project. Kanban, coming from japanese and meaning signboard or billboard 
\footcite{what-is-kanban} is a form of project management, that chooses to use continuous flow of tasks instead of splitting them up into sprints, like SCRUM. 
It also lacks distinct roles for team members, that would be defined in SCRUM (e.g. a SCRUM-Master), and uses cycle-time \footcite{cycle-time-lead-time},
instead of velocity as a metric for performance. 

\subsubsection{Structure of RECT's Kanban Board}
RECT's Kanban-Board is hosted on trello, because it offers easy setup and operation. The board consists of the following columsn:

\begin{itemize}
\item A "Backlog" column for tasks, that are to be done when there is capacity.
\item A "Todo" column for tasks, that need to be done soon.
\item A "Design" column for tasks, that need to have a specific design (e.g. the Architecture of a TCP-Service) and for thinking about tests, that need to be written.
\item A "Doing" column for tasks, that are currently being worked on. This has a limit of 4 tasks, so each member can only work on one thing.
\item A "Testing" column for task, that have been roughly implemented, but still need some changes to pass all tests.
\item A "Code Review" column for tasks, that have been thoroughly tested and need to be reviewed by a member in order to be merged.
\item A "Done" column for tasks, that have been tested, reviewed and merged with the main branch.
\end{itemize}

\subsection{Meetings with the projects supervisor}
There were weekly meetings that were held with the diploma theses' supervisor Harald Haberstroch on fridays, where the progress of the project and upcoming tasks were discussed.
The members of the team could also request feedback on their work and get help, or advice for problems they may have been having.

\subsection{Hours spent outside of school hours} 
Apart from these weekly meetings the members of the team met regularily on wednesdays and thursdays after school to work on the project together and discuss problems, 
that came up between the weekly meetings.

\subsection{Version Control}
Version control was done with the industry standard version control system git and the project was hosted on \href{https://gitlab.htlwrn.ac.at/Sztavinovszki.Jeremy/RECT}{the schools gitlab server}, 
which also contained submodules hosted on \href{https://github.com/F-WuTS/}{F-WuTS} and on \href{https://github.com/if-loop69420}{Jeremy Sztavinovszki's personal github account}.

\tocdata{toc}{$\rightarrow$\textit{Jeremy Sztavinovszki}}
\section{Outline}
\textbf{Author: Jeremy Sztavinovszki}
This "Introduction" is followed by a "Study of Literature", where the authors describe which books were read, why they were read and what knowledge and insights were gained.
After the "Study of Literature" is the "Methods" chapter. This chapter covers the technologies and languages used, why they were chosen and seeks to provide a basic understanding of the 
technologies. Following that chapter is the
"Implementation" chapter, which covers the implementation details of each of the parts and explains design choices and tradeoffs taken. The last two chapters are the
"Experiments" chapter and the "Conclusion" chapter. The "Experiments" chapter goes over the experiments performed for the thesis and their results. The "Conclusion" chapter
recaps the contents of the thesis and draws a conclusion regarding the thesis' success and findings.

\filbreak
