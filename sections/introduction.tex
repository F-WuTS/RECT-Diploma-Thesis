\chapter{Introduction}

\makeatletter
\renewcommand\paragraph{\@startsection{paragraph}{4}{\z@}%
% display heading, like subsubsection
                                     {-3.25ex\@plus -1ex \@minus -.2ex}%
                                     {1.5ex \@plus .2ex}%
                                     {\normalfont\normalsize\bfseries}}
 \setcounter{secnumdepth}{4}
\makeatother

\vspace{2mm}

\section{Motivation}
\textbf{Author: Sztavinovszki}

In almost every robotics application nowadays you need some kind of communication. Wether it is a robot communicating
its data to a home-base, or two robots sharing data with one another. Over the past years communication has drastically
improved with new protocols and technology, such as Bluetooth Low energy. On the other hand there are still people, that
don't use any of the communication technologies described before, because most of the established frameworks seem too
complex or have too many requirements regarding performance. This 

\section{Goal}
\textbf{Author: Sztavinovszki}

This diploma thesis will take explore how to write a real-time communication framework using new technologies, such as BLE with TCP
and UDP.

At the end of the project the framework should be useable for sending data between two KIPR Wombat-Controllers. It should be capable
to send well structured data, e.g. a JSON-file, as well as streaming data, e.g. a series of pictures, between the robots. All sent data should have the
option to be compressed with different compression formats ranging from lossless to lossy formats. When using protocols, that don't make guarantees about
the completeness of the received data RECT will not provide extra safeguards to make sure all sent data is also received.
RECT should be able to provide a stable and fast connection at reasonable distances for the selected protocol (e.g. circa 30 meters of line-of-sight, for BLE).
RECT should provide libraries for the languages Python and C++ that hide the complexity of communication and provide nice abstractions simplifying its use.
For this to be made possible the following requirements have to be met:
\begin{itemize}
\item A Rust library for communicating via TCP, BLE and UDP has to be written.
\item A Rust gRPC service using the communication library has to be implemented.
\end{itemize}

\section{History}
\textbf{Author: Dragosits}
\subsection{Networks}
The history of communication and coordination between seperate systems is long and extensive. Nowadays it is an essential 
function of almost every technological device. In the field of computing it started in the late 1950s with SAGE (Semi-Automatic Ground Enviornment),
which was a network of computers and networking technology created by the United States of America military, 
and it allowed the transfer of radar data nation-wide.\footcite[][89]{A_New_History_of_Modern_Computing}\\
The next major step forward was the beginning of ARPANET in 1969, which served as a connection between multiple north american 
universities, and laid the groundwork for the modern internet.\footcite[][25]{How_the_web_was_born}
In more recent times the Internet of Things is commonplace, and is used for the interchange of terabytes of data each day. 
But there are also many smaller private networks used either for simple processes or sensitive data, that shouldn't be 
accesible by a theoretical viewer outside the trusted circle. 

\subsection{Protocols}
In the current network communication landscape, various protocols enable the smooth transfer of data. One of these protocols is the 
Transmission Control Protocol (TCP), which has become the dominant choice among Transport layer protocols due to its widespread use on the internet. 
The TCP/IP protocol, which was first described in RFC 675\footcite{rfc_675}, is the foundation of modern network communication. It establishes a robust framework that 
underpins the vast expanse of the internet. Alongside TCP, the User Datagram Protocol (UDP), conceived by David Patrick Reed in 1980, is a formidable companion. 
These protocols, TCP and UDP, have solidified their positions as stalwarts in data transfer methodologies across the internet and analogous networks.\\

The introduction of TCP/IP was a significant moment in the development of network communication. RFC 675 established the basic principles that continue 
to influence modern internet protocols. UDP, although different in design and use, complements TCP by offering a lightweight option for situations where 
real-time data transmission is more important than guaranteed delivery. TCP, UDP, and IP collectively form the Internet Protocol Suite, which governs the 
functioning of the internet and facilitates global connectivity. Their enduring significance in shaping network protocols underscores their indispensable role 
in our interconnected digital world.

\subsection{Bluetooth}
Establishing seamless communication between multiple mobile and fixed devices in close proximity, along with the formation of Personal Area Networks (PANs), 
has traditionally presented considerable challenges. To address this issue, Bluetooth technology emerged as a groundbreaking solution. The genesis of this 
innovative concept can be traced back to 1989 when Ericsson Mobile envisioned a wireless solution for headsets. Following the conceptualization phase, active 
development began in 1994. In 1997, IBM collaborated with Ericsson to integrate Bluetooth technology into the IBM ThinkPad, recognizing its potential. 
This collaboration resulted in the incorporation of Bluetooth functionality into both the ThinkPad notebook and an Ericsson mobile phone, marking a pivotal moment 
in the evolution of wireless connectivity.\\

In 1999, the first device equipped with Bluetooth functionality, a wireless headset, marked a significant leap forward for the technology. This milestone 
revolutionised the way devices communicate over short distances and laid the foundation for the widespread adoption of Bluetooth across a myriad of electronic 
devices, ranging from smartphones to smart home gadgets. The evolution of Bluetooth from its origins as a wireless headset solution to its current status as a 
crucial element of various technological ecosystems highlights its lasting influence in enabling effortless connectivity in both personal and professional contexts.

\section{Project Management}
\textbf{Author: Sztavinovszki}

\subsection{Kanban}
In a project like RECT, that is relatively small, but still complex and needs the members to be able to work independently project management is of utmost importance. Today the most
obvious choice for working on a group project is using SCRUM \footcite{what-is-scrum}, because it allows complicated projects to be completed in a fashion, 
that is able to adapt to new requirements by for example customers. One of the biggest downsides of SCRUM the members of the project saw was the high overhead of planning sprints. 
For that reason Kanban was chosen for managing the tasks, that need to be done in order to finish the project. Kanban, coming from japanese and meaning signboard or billboard 
\footcite{what-is-kanban} is a form of project management, that chooses to use continuous flow of tasks instead of splitting them up into sprints, like SCRUM. 
It also lacks distinct roles for team members, that would be defined in SCRUM (e.g. a SCRUM-Master), and uses cycle-time \footcite{cycle-time-lead-time},
instead of velocity as a metric for performance. 

\subsubsection{Structure of a Kanban Board}
A Kanban-Board is made up of several columns. The most common columns used are:
\begin{itemize}
\item Backlog
\item Todo
\item Doing
\item Done
\end{itemize}
These columns contain anywhere from zero to infinite cards, while some have a limit imposed on how many cards are allowed in a column at a given point in time.

\subsubsection{Life-Cycle of a card}
At the beginning of a cards life-cycle the card is given a name and a description of a task. It is then put into the "Backlog" column, where it will stay until there are capacities
to work on the given task. Once a developer starts working on a card it is put into the "Doing" column of the board, until the task has been completed. It is then either put into
some kind of "Testing", or "Quality Control" column, or it goes straight into done. Reaching the "Done" column of a Kanban board also means the card has reached the end of its
life-cycle.

\subsubsection{History of Kanban}
Kanban was developed by a japanese Engineer named Taiichi Ohno while working at Toyota Motor Corporation to improve the efficiency of manufacturing processes. Although it was originally
designed to solve problems in manufacturing processes the principles of Kanban can also be applied to software development. These 4 principles are as follows:

\paragraph{Visualize Work}
One of the most challenging tasks in project management is understanding the current progress and status of a project. To make this easier Kanban emplores a board and cards with tasks to
visualize progress, bottlenecks and the status quo of a project.

\paragraph{Using a Pull Model}
When working on a project stakeholders often tend to push more and more work onto the developers, which leads to stress and poor quality. To remedy this Kanban defines a certain level of
quality, that needs to be met in order for a task to be considered done. Instead of pushing extra work onto developers the stakeholders simply add tasks to the backlog and let the 
developers implement them as soon as they have the capacity to do so.

\paragraph{Enforce a limit on ongoing tasks}
Often teams struggle with working on too many things at once. This leads to reduced performance and velocity, because the context of the different tasks is constantly switched. 
The solution Kanban offers to this problem is imposing a limit on how many tasks can be in a column of the Kanban Board at once. 

\paragraph{Measuring CI}
In order to achieve Continuous Improvement (CI) in a project teams need a way to measure the effectiveness of their workflows. The dynamic view of the states of work in a workflow, that is 
provided by Kanban makes it easy for teams to experiment with how they do processes and see how they impact their productivity. Avid users of Kanban use measurements like lead-time 
cycle-time \footcite{cycle-time-lead-time} for Continuous Improvement.  

\subsubsection{Structure of RECT's Kanban Board}
RECT's Kanban-Board is hosted on trello, because it offers easy setup and operation. The board consists of the following columsn:

\begin{itemize}
\item A "Backlog" column for tasks, that are to be done when there is capacity.
\item A "Todo" column for tasks, that need to be done soon.
\item A "Design" column for tasks, that need to have a specific design (e.g. the Architecture of a TCP-Service) and for thinking about tests, that need to be written.
\item A "Doing" column for tasks, that are currently being worked on. This has a limit of 4 tasks, so each member can only work on one thing.
\item A "Testing" column for task, that have been roughly implemented, but still need some changes to pass all tests.
\item A "Code Review" column for tasks, that have been thoroughly tested and need to be reviewed by a member in order to be merged.
\item A "Done" column foor tasks, that have been tested, reviewed and merged with the main branch.
\end{itemize}

\subsection{Meetings with the projects supervisor}
There were weekly meetings that were held with the diploma theses' supervisor Harald Haberstroch on fridays, where the progress of the project and upcoming tasks were discussed.
The members of the team could also request feedback on their work and get help, or advice for problems they may have been having.

\subsection{Hours spent outside of school hours} 
Apart from these weekly meetings the members of the team met regularily on wednesdays and thursdays after school to work on the project together and discuss problems, 
that came up between the weekly meetings.

\subsection{Version Control}
Version control was done with the industry standard version control system git and the project was hosted on \href{https://gitlab.htlwrn.ac.at/Sztavinovszki.Jeremy/RECT}{the schools gitlab server}, 
which also contained submodules hosted on \href{https://github.com/F-WuTS/}{F-WuTS} and on \href{https://github.com/if-loop69420}{Jeremy Sztavinovszki's personal github account}.

\section{Outline}
\textbf{Author: Sztavinovszki}
Beginning with with the Technology section the thesis describes the different libraries, languages and technologies used and explains why the respective technologies have been chosen for RECT.
After the Technology section there's a dive into the implementation details and design rationale behind the materialization of the different practical parts of the project.
The Implementation section will also include comparisons between the different ways of implementing certain components of RECT from a standpoint of readability and maintainability, but not of performance,
which will be discussed in the Testing section of this thesis. As stated, the Testing section of the thesis will include comparisons in performance of different implementations,
technologies and field tests of the technology resulting from RECT.

%\section{Section}
%More text. \lipsum[1] See Figure~\ref{pic:example}.

%\begin{figure}[h]
%	\centering
%	\includegraphics[width=2.5in]{img/example.png}
%	\caption{Picture description.}
%	\label{pic:example}
%\end{figure}

%\subsection{Subsection}
%\lipsum[1]

%\subsection{Subsection}
%\lipsum[1] See Table~\ref{tab:example}.

%\begin{center}
%	\begin{tabular}{| l | l | l |}
%		\hline
%		\bfseries Header 1 & \bfseries Header 2 & \bfseries Header 2 \\
%		\hline
%		Text & text & text \\
%		\hline
%		Text & text & text  \\
%		\hline
%		Text & text & text  \\
%		\hline
%	\end{tabular}
%	\label{tab:example}
%\end{center}

%\lipsum[1] Some references can be found at \footcite{robo4you} or at \footcite{Hope_Learning_TensorFlow}.
%

\filbreak
